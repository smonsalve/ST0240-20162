\section{Problema 2: INVCNT - Inversion Count}

Link Problema \url{http://www.spoj.com/EAFIT201/problems/INVCNT/}

\subsection{Descripción}

Sea un Arreglo de \emph{n} eneteros positivos $A[0...n-1]$. Si $ i \< j$ y $ A[i] \> A[j]$ entonces el par (i, j) es llamado una inversión de A.
Dado n y un Arreglo A su tarea es encontrar la cantidad de inversiones de A.

\subsubsection{Entrada}
La primera linea contiene un entero t, la cantidad de casos a procesar cada caso seguido de una linea en blanco.
Cada uno de los casos comienza con un umero $ n \geq 200000 $ luego n+1 lineas siguen.
En la \emph{i-esima} linea un numero $A[i - 1]$ es dado $(A[i - 1] \leq 10^7).$ La linea \emph{(n + 1)} es una linea en blanco
m

\subsubsection{Salida}

Para cada prueba retorne una linea dando el numero de inversiones en A

\subsection{Ejemplo}

\begin{tabu} to \linewidth {|X[c]|X[c]|}
  \hline
  \rowfont{\bfseries\itshape\large} Entrada & Salida \\
  \hline
  \lstinputlisting{inv.in} & \lstinputlisting{inv.ot} \\
  \hline  
\end{tabu}
