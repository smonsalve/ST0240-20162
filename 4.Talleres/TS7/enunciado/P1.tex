 \section{Descripcion}

Un profeso decidio hacer una actividad evaluativa que consistia de 3 problemas diferentes, y le pidio a sus alumnos que respondieran 2, si respondian 3 les otorgaba una bonificación.

Para ello usted debe realizar un programa que dado unos archivos de entrada calcule la nota correspondiente a cada alumno, para ello se debe cumplir lo siguiente:\\

Se tiene un archivo llamado \textbf{Nombres.txt} con una lista de archivos a procesar. Cada archivo tiene dos lineas, la primera un identificador y la segunda un arreglo de caracteres, donde hay cuatro numeros separados por espacios, y cada elemento de ese arreglo corresponde al numero de problemas enviados en un dia determindado.\\
        
Son 3 problemas, 2 normales y uno de bonificación, dependiendo del día en que envie los primeros dos estos tienen cierta calificación:
Si envió 1 o 2 problemas dependiendo del día, la calificacion tendrá entonces el siguiente valor:\\

Martes: 2.5 por cada problema \\
Miércoles: 2.0 por cada problema \\
Jueves: 1.5 por cada problema \\ 
Viernes: 0 por cada problema \\


Si envió mas de 2  problemas independientemente del día que se envie la bonificación esta tiene el valor de 1 uxnidad.\\

Restricción: La suma de los problemas enviados en todos los días no puede superar 3, ni un dia puede tener mas de 3 problemas enviados.


\newpage

\section{Entrada}

Para su problema se le entregara una carpeta de entrada como la que se envia adjunta, En esa carpeta esta el archivo indice de \textbf{Nombres.txt} donde estan los nombres de los archivos a procesar.

Tambien encontraran allí los correspondientes archivos tipo \textbf{XXXXXYYYYWWW.txt} con la información a procesar.

ej:\\

%\lstinputlisting[caption=My caption]{sourcefile.lang}
\lstinputlisting[caption=Nombres.txt]{../input/Nombres.txt}
\lstinputlisting[caption=000000000001.txt]{../input/000000000001.txt}

\newpage

\section{Salida}
        
Su objetivo es calcular todas las notas y dado un identificador entregar la nota que saco tal persona en la actividad evaluativa.\\

Su programa debe generar un archivo de notas llamado \textbf{Notas.ot} con el siguiente formato:

\lstinputlisting[caption=Notas.ot]{../input/Notas.ot}


\newpage

\section{Rubrica Evaluativa}


\begin{tabular}{|l|c|}
  \hline
  \textbf{Descripción} & \textbf{Porcentaje}\\
  \hline
  Leer Archivo de nombres de archivo (indice) & 15\%\\
  \hline
  Leer Archivo de notas de cada estudiante  & 15\%\\
  \hline
  Manejar correctamente una excepción en caso de que un arhivo no tenga el formato adecuado & 15\% \\
  \hline
  Manejar correctamente una excepción en caso de que un archivo no exista & 15\%\\
  \hline
  Calcular las notas de todos los estudiantes (indice) & 15\%\\
  \hline
  Dado un identificador retornar la nota de ese estudiante & 15\%\\
  \hline
  Generar un archivo con las notas de cada estudiante & 10\%\\
  \hline
\end{tabular}

