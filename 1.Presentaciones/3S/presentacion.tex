\documentclass[xcolor=svgnames]{beamer} 
\usefonttheme[onlymath]{serif}
\usepackage{tcolorbox}
\usepackage{multimedia}
\usepackage{listings}
\usepackage[utf8]{inputenc}
\usepackage[T1]{fontenc}
\usepackage[spanish]{babel}
\usepackage{mathtools}  % amsmath improved
%\usepackage{amsmath}
\usepackage{amssymb}
\usepackage{amsthm}
\usepackage{cancel}
\usepackage{multimedia}
\usepackage{hyperref}
\usepackage{algorithm2e}
%\usepackage{algorithm}
%\usepackage{algorithmic}
\usepackage{xcolor}
%\usepackage{pgfplots}
\usepackage{booktabs}   % Fancy table borders.

\newcommand{\grn}[1]{\textcolor{Green}{#1}}
\newcommand{\red}[1]{\textcolor{Red}{#1}}
\newcommand{\ple}[1]{\textcolor{Purple}{#1}}
\newcommand{\ong}[1]{\textcolor{DarkGoldenrod}{#1}}

\usetheme{Warsaw}

\definecolor{eafitColor}{rgb}{0.03515625,0.30078125,0.55078125}
\definecolor{eafitColor2}{rgb}{ 0.30859375,  0.5234375 ,  0.6953125}
\definecolor{mygreen}{rgb}{0.0 ,0.6, 0.0}

\usecolortheme[named=eafitColor]{structure} 
\setbeamercolor*{palette primary}{use=structure,fg=white,bg=eafitColor}
\setbeamercolor*{palette quaternary}{fg=white,bg=eafitColor2!90!eafitColor2}

\defbeamertemplate*{footline}{shadow theme}
{%
  \leavevmode%
  \hbox{\begin{beamercolorbox}[wd=.5\paperwidth,ht=2.5ex,dp=1.125ex,leftskip=.3cm plus1fil,rightskip=.3cm]{author in head/foot}%
    \usebeamerfont{author in head/foot}\hfill\insertshortauthor
  \end{beamercolorbox}%
  \begin{beamercolorbox}[wd=.5\paperwidth,ht=2.5ex,dp=1.125ex,leftskip=.3cm,rightskip=.3cm plus1fil]{title in head/foot}%
    %\usebeamerfont{title in head/foot}\insertshorttitle\hfill\insertframenumber\,/\,\inserttotalframenumber%
    \usebeamerfont{title in head/foot}\insertshorttitle\hfill\insertframenumber\,/\,\inserttotalframenumber%
  \end{beamercolorbox}}%
  \vskip0pt%
}
%\setbeamertemplate{footline}{\hfill\insertframenumber/\inserttotalframenumber} 

\everymath{\color{blue}}
\everydisplay{\color{blue}}

\newcommand{\highlight}[1]{%
  \colorbox{yellow!50}{$\displaystyle#1$}}
  
\newcommand{\mathColor}[1]{{\color{blue}#1}}
\newcommand{\emphRed}[1]{{\color{red}#1}}

%~~~~~~~~~~~~~~~~~~~~~~~~~~~~~~~~~~~~~~~~~~~~~~~~~~~~~~~~~~~~~~~~~~~~~~~~~~
%                           Example environment

 \newtheoremstyle{example}{\topsep}{\topsep}%
     {}%         Body font
     {}%         Indent amount (empty = no indent, \parindent = para indent)
     {\bfseries}% Thm head font
     {}%        Punctuation after thm head
     {\newline}%     Space after thm head (\newline = linebreak)
     {\thmname{#1}\thmnumber{ #2}\thmnote{ #3}}%         Thm head spec

   \theoremstyle{example}
%   \newtheorem{example}{Example}[chapter]


%\AtBeginSection{\frame{\sectionpage}}
%\AtBeginSubsection{\frame{\subsectionpage}}

%~~~~~~~~~~~~~~~~~~~~~~~~~~~~~~~~~~~~~~~~~~~~~~~~~~~~~~~~~~~~~~~~~~~~~~~~~~~~~~~

\graphicspath{{../imagenes/},{figures2/}}

\logo{\includegraphics[width=1.5cm]{../aux/LogoAzul.png}\vspace{-.2cm}} 


\title[ST0240-063]{Programación de Computadores }
\subtitle{Semana 3: Viernes 5 de Agosto de 2016}
\author{
  Sergio Monsalve \\
  smonsal3@eafit.edu.co
}
\institute{
  Departamento de Informática y Sistemas \\
  Universidad EAFIT, Medellin, Colombia\\\vspace{0.5cm}
}
\date{Julio, 2016}

\makeatletter
    \newenvironment{withoutheadline}{
        \setbeamertemplate{headline}[default]
        \def\beamer@entrycode{\vspace*{-\headheight}}
    }{}
\makeatother

%~~~~~~~~~~~~~~~~~~~~~~~~~~~~~~~~~~~~~~~~~~~~~~~~~~~~~~~~~~~~~~~~~~~~~~~~~~
% Comment this line to remore the table of contents on top of the slide

%\setbeamertemplate{headline}{}

\beamertemplatenavigationsymbolsempty
%~~~~~~~~~~~~~~~~~~~~~~~~~~~~~~~~~~~~~~~~~~~~~~~~~~~~~~~~~~~~~~~~~~~~~~~~~~

\begin{document}

\setbeamertemplate{background canvas}{
%	    \includegraphics[width=1.1\textwidth]{region}
}
	
\begin{frame}[plain]
	\titlepage
\end{frame}
\setbeamertemplate{background canvas}{}
%\maketitle 

%~~~~~~~~~~~~~~~~~~~~~~~~~~~~~~~~~~~~~~~~~~~~~~~~~~~~~~~~~~~~~~~~~~~~~~~~~~

%\section*{Outline}
\begin{withoutheadline}
  \begin{frame}
    \setcounter{tocdepth}{1}
    \frametitle{Contenido}     
    \tableofcontents
  \end{frame}
\end{withoutheadline}


\section{Introducción}

\subsection{Pre - clase}

\begin{frame}
  \frametitle{Pendientes}
  \begin{itemize}
  \item Quiz 1\%
  \item Dudas y Preguntas
  \item Libro: Revisión de Capítulos 1,2,3,4
  \end{itemize}
\end{frame}

\begin{frame}
  \frametitle{Clase Anterior}
  \begin{itemize}
    \item Expresiones
    \item Condiciones
    \item Ciclos
  \end{itemize}
\end{frame}


\begin{frame}
  \frametitle{Revisión Capitulos 1,2,3,4 del libro}
  \begin{itemize}
  \item Cap1
  \item Cap2
  \item Cap3
  \item Cap4
  \end{itemize}
\end{frame}

\section{Contenido}

\begin{frame}
  \frametitle{Para hoy...}
\begin{itemize}
\item Condiciones: If,If-else, switch
\item Ciclos: while, do-while, for, for.each
\end{itemize}
\end{frame}

\begin{frame}
  \frametitle{test}
  \lstinputlisting[language=Python]{Prueba.py}
\end{frame}

\section{Cierre}
\subsection{post-clase}

\begin{frame}
  \frametitle{Pendientes para la próxima clase}
  \begin{itemize}
  \item{Libro guia: Capitulo }
    \item{SPOJ: problema de estudio}
  \item{Quiz 1\%}
  \item{Taller 5\%}
  \end{itemize}
\end{frame}

\subsection{Enlaces}
\begin{frame}
  \frametitle{Recursos Adicionales}
  \begin{itemize}
  \item https://coderbyte.com/course/learn-python-in-one-week
  \end{itemize}
\end{frame}

\begin{frame}
  \frametitle{Pendientes para la próxima clase}
  \begin{itemize}
  \item{Libro guia: Capitulos }
    \item{SPOJ: problema }
  \item{Quiz 1\%}
  \item{Evaluación? No, pero preparación para la proxima!}
  \end{itemize}
\end{frame}

\subsection{Referencias}

\begin{thebibliography}{9}
\bibitem{farrell}
  Joyce Farrell,
 \emph{Introducción a la programación: lógica y diseño},
 Thomson Learning,2001.

\bibitem{Gries}
  Gries, Paul and Campbell, Jennifer and Montojo, Jason,
  \emph{Practical Programming: An Introduction to Computer Science Using Python 3},
  Pragmatic Bookshelf
  2013
 
\end{thebibliography}

\end{document}

%\begin{frame}
%  \frametitle{}
%\end{frame}
 
%\begin{frame}
%  \frametitle{}
% \begin{itemize}
% \item
% \end{itemize}
%\end{frame}
