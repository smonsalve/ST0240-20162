\documentclass[11pt,letterpaper]{article}
\usepackage{samstyle}

\title{Ejercicios de programación}
        
\author{
	Profesores:\\[0.5cm]
	Sergio Andrés Monsalve Castañeda\\
	smonsal3@eafit.edu.co\\[1cm]
    John Jairo Silva Zuluaga \\
    jsilvazu@eafit.edu.co
}

\begin{document}
 
\pagestyle{fancyplain}
\fancyhf{}
\headheight=20pt %para cambiar el tamaño del encabezado
\renewcommand{\headrulewidth}{0pt} %espesor del encabezado

% \lhead %la "L" indica a la izquierda
% {
% }

\fancyfoot[c]{\thepage}

\maketitle

%\begin{minipage}{3cm}
% \includegraphics[width=15cm]{aux/SamCanny.jpg}
%\end{minipage}

 \section{Descripcion}

Un profeso decidio hacer una actividad evaluativa que consistia de 3 problemas diferentes, y le pidio a sus alumnos que respondieran 2, si respondian 3 les otorgaba una bonificación.

Para ello usted debe realizar un programa que dado unos archivos de entrada calcule la nota correspondiente a cada alumno, para ello se debe cumplir lo siguiente:\\

Se tiene un archivo llamado \textbf{Nombres.txt} con una lista de archivos a procesar. Cada archivo tiene dos lineas, la primera un identificador y la segunda un arreglo de caracteres, donde hay cuatro numeros separados por espacios, y cada elemento de ese arreglo corresponde al numero de problemas enviados en un dia determindado.\\
        
Son 3 problemas, 2 normales y uno de bonificación, dependiendo del día en que envie los primeros dos estos tienen cierta calificación:
Si envió 1 o 2 problemas dependiendo del día, la calificacion tendrá entonces el siguiente valor:\\

Martes: 2.5 por cada problema \\
Miércoles: 2.0 por cada problema \\
Jueves: 1.5 por cada problema \\ 
Viernes: 0 por cada problema \\


Si envió mas de 2  problemas independientemente del día que se envie la bonificación esta tiene el valor de 1 uxnidad.\\

Restricción: La suma de los problemas enviados en todos los días no puede superar 3, ni un dia puede tener mas de 3 problemas enviados.


\newpage

\section{Entrada}

Para su problema se le entregara una carpeta de entrada como la que se envia adjunta, En esa carpeta esta el archivo indice de \textbf{Nombres.txt} donde estan los nombres de los archivos a procesar.

Tambien encontraran allí los correspondientes archivos tipo \textbf{XXXXXYYYYWWW.txt} con la información a procesar.

ej:\\
\lstinputlisting{../input/Nombres.txt}
\lstinputlisting{../input/000000000001.txt}

\newpage

\section{Salida}
        
Su objetivo es calcular todas las notas y dado un identificador entregar la nota que saco tal persona en la actividad evaluativa.\\

Su programa debe generar un archivo de notas llamado \textbf{Notas.ot} con el siguiente formato:

\lstinputlisting{../input/Notas.ot}


\newpage

\section{Rubrica Evaluativa}


\begin{tabular}{|c|c|}
  \hline
  \textbf{Descripción} & \textbf{Porcentaje}\\
  \hline
  Leer Archivo de nombres de archivo (indice) & 15\%\\
  \hline
  Leer Archivo de notas de cada estudiante  & 15\%\\
  \hline
  Manejar correctamente una excepción en caso de que un arhivo no tenga el formato adecuado & 15\% \\
  \hline
  Manejar correctamente una excepción en caso de que un archivo no exista & 15\%\\
  \hline
  Calcular las notas de todos los estudiantes (indice) & 15\%\\
  \hline
  Dado un identificador retornar la nota de ese estudiante & 15\%\\
  \hline
  Generar un archivo con las notas de cada estudiante & 10\%\\
  \hline
\end{tabular}



\begin{enumerate}
        \item A partir de 5 números, obtener su promedio y mostrar el entero mas cercano por encima. 
        Ejemplo: para 10, 8, 14, 2, 3 el promedio es 7.4, se mostraría el número 8.
        
        \item a partir de una lista de valores(vector o lista), hallar el promedio y la desviación estándar.
        
        \item Leer dos puntos en el espacio cartesiano, decir cual es la distancia entre ellos.
        
        \item Calcular la combinatoria de M y N, si $\dbinom{M}{N}$ = $\frac{M!}{N!(M-N)!}$
        
        \item Desarrolle una función en la cuál dados dos números (m y n), imprima un rectangulo de n x m '*' 
        
        \item Implemente una funcion que calcule Área y perímetro de un rectángulo (dos parámetros)
        
        \item Implemente una funcion que calcule Área y perímetro de un triángulo (3 parámetros)
        
        \item Implemente una funcion que calcule Área y perímetro de un círculo (se pasan 1 parámetro)
        
        \item Implemente una funcion que calcule Volumen y área de un cilindro( 2 parámetros)
        
        \item Dados 'n' números, imprimir, la suma del mayor y del menor
        
        \item Escribir un programa que diga si un número es capicúa (Un número capicúa es el que se puede leer igual al derecho y al revés)(Hacerlo sin transformar a cadena de caracteres)
        
        \item Pide por teclado un número entero positivo (debemos validarlo) y muestra  el número de cifras que tiene. Por ejemplo: si introducimos 1250, nos muestra que tiene 4 cifras.
        
        \item Dado un número mayor que 0 que representa cuantidad de segundos retornar cuantas horas, minutos y segundos hay, ejemplo: para el número 15723 imprimirá: 4 horas, 22 minutos y 3 segundos.
        
        \item Del anterior, hacer lo contrario, es decir, dados 3 valores(horas, minutos y segundos), escribir el número de segundos.
        
        \item Dados dos números: si los números son positivos, restarlos, si son negatívos, multiplicarlos, si son diferentes, dividirlos.
        
        \item Dados N números, imprimirlos en orden descendente
        
        \item Dados dos números enteros positivos N y D, se dice que D es un divisor de N si el residuo de dividir N entre D es 0. Se dice que un número N es perfecto si la suma de sus divisores (excluido el propio N) es N. Por ejemplo 28 es perfecto, pues sus divisores (excluido el 28) son: 1, 2, 4, 7 y 14 y su suma es 1+2+4+7+14=28. Hacer un programa que dado un número N nos diga si es o no perfecto.
        
        \item Un año es bisiesto si es múltiplo de 4, exceptuando los múltiplos de 100, que sólo son bisiestos cuando son múltiplos además de 400, por ejemplo el año 1900 no fue bisiesto, pero el año 2000 si lo será. Hacer un programa que dado un año A nos diga si es o no bisiesto.
        
        \item Hacer un programa que dados dos números(los catetos de un triángulo rectángulo), entregue el valor de la hipotenusa.
        
        \item A partir de 3 números naturales, decir si estos corresponden a un triángulo equilátero, isósceles o escaleno.
        
        \item Escribir los números que son múltiplos de 3 y de 7, del 1 al 1000

    \end{enumerate}
%\bibliographystyle{plainnat}
%\bibliography{refs}

\end{document}


