\documentclass[11pt,letterpaper]{article}
\usepackage{samstyle}

\title{Daño en base de datos}
        
\author{
	Profesor\\
	Sergio Andres Monsalve Castañeda\\
	smonsal3@eafit.edu.co
}

\begin{document}
 
\pagestyle{fancyplain}
\fancyhf{}
\headheight=20pt %para cambiar el tamaño del encabezado
\renewcommand{\headrulewidth}{0pt} %espesor del encabezado

% \lhead %la "L" indica a la izquierda
% {
% }

\fancyfoot[c]{\thepage}

\maketitle

\begin{minipage}{3cm}
% \includegraphics[width=15cm]{aux/SamCanny.jpg}
\end{minipage}


\section{Descripción}

Resulta que en una base de datos de la universidad se daño una de las columnas de la tabla que contenia los estudiantes de un grupo. En tal tabla había 3 estudiantes con el mismo nombre y se borro tal nombre.

\section{Objetivo}

Nuestro objetivo es recuperar la integridad de la base de datos.

\section{Entrada}

Cuando la universidad nos diga cual fue el nombre que fue borrado nuestro programa deberá leer tal nombre como entrada.

\section{Salida}

Se espera que su programa entregue como salida (Resultado) las siguientes lineas:

Nombre + espacio +  Apellido1 + espacio + codigo1 \\
Nombre + espacio +  Apellido2 + espacio + codigo2 \\
Nombre + espacio +  Apellido3 + espacio + codigo3 \\

\section{Ejemplo}
\subsection{Entrada}
\lstinputlisting{db.in}
\subsection{Salida}
\lstinputlisting{db.ot}

%\bibliographystyle{plainnat}
%\bibliography{refs}

\end{document}


